\documentclass[12pt, a4paper]{article}
\usepackage{titlesec}
\usepackage{bm}

\titleformat{\subsection}[runin]
{\normalfont\large\bfseries}{\thesubsection}{1em}{}
\begin{document}

\title {\textbf {CSE 551
\\Assignment 1}}
\author {Arun Kumar Kumarasamy}
\date{\today}
\maketitle

\section*{Question 1}
 Prove or disprove the following assertions:\\ 
\\(i) If $f(n) =O(g(n))$ then $\log_2 f(n) =O(\log_2 g(n))$
\\(ii) If $f(n) =O(g(n))$ then $3^{f(n)}=O(3^{g(n)})$
\\(iii) If $f(n) =O(g(n))$ then $f(n)^{3}=O(g(n)^{3})$ 
\section*{Solution:}
\subsection*{(i)} \textbf{False}. Disprove by counterexample.
\\Since $f(n) = O(g(n))$, we know that, $$ 0 \leq  f(n) \leq C * O(g(n)) $$
\\Suppose $f(n) = 2$ and $g(n) = 1$, Then for all n,  $2 \leq C * 1$ for any constant $C \geq 2$. However, $\log f(n)$ is not in $O(\log g(n))$ in the case that $f(n) = 2$ and $ g(n) = 1$  since $ \log 2 > C * \log (1) = 1 > C * 0$ for any n and any constant C.\\\\\\\\\\\\
\subsection*{(ii)} \textbf{False}. Disprove by counterexample.
\\Since $f(n) = O(g(n))$, we know that, $$ 0 \leq  f(n) \leq C * O(g(n)) $$
\\Suppose $f(n) = 3n$ and $g(n) = n$. Then for all n, $3n \leq C * n$ for any constant $C \geq 3$. However, $3^{3n}$ is not in $O(3^{n})$ in the case if $f(n) = 3n$ and $g(n) =n$ because $3^{3n} >> C*3n$ implies that, $1>C*0$ for any $n>0$ and any constant C. Thus $3^{f(n)} > O(3^{g(n)})$
\subsection*{(iii)} \textbf{True}
\\Since $f(n) = O(g(n))$, we know that, $$ 0 \leq  f(n) \leq C * O(g(n)) $$
Hence there exists some $n_0$ for All $n>n_0$, such that $f(n) \leq C * g(n)$ for some C. Hence for all $n>0$, $C>0$, $f(n)^{3} \leq C^{3} * g(n)^{3}$
\section*{Question 2}
Algorithm $A_1$ takes $10^{-4} \times 2^{n}$ seconds to solve a problem instance of size n and Algorithm $A_2$ takes $10^{-2} \times n^{3}$ seconds to do the same on a particular machine.\\(i) What is the size of the largest problem instance A2 will be able solve in one year ?\\(ii) What is the size of the largest problem instance A2 will be able solve in one year on a machine one hundred times as fast ?\\(iii) Which algorithm will produce results faster, in case we are trying to solve problem instances of size less than 20 ?
\section*{Solution:}
\subsection*{(i)} \textbf{1466}. \\$A_2$ takes $10^{-2} \times n^{3}$ seconds to solve a problem instance of size n
\\Taking 1 year = 365 days = 31536000 seconds,  \\size for one year is given by $$ 10^{-2} \times n^{3} = 31536000$$ hence $n = \sqrt[3]{31536000}$\\ \textbf{n = 1466}
\subsection*{(ii)} \textbf{6806} \\Again considering 1 year = 365 days = 31536000 seconds, \\let $N^{'}$ be the size of problem instance solved by 100x machine. \\Time taken by 100x machine to solve $N^{'}$ sized problem is given by, $$ 10^{-2} \times N^{'^{3}}$$ which is 100 times the time taken by regular machine. Hence, $$ 10^{-2} \times N^{'^{3}} = 100 \times 31536000$$ Solving for $N^{'}$ gives us, $\bm{ N^{'} = 6806}$
\subsection*{(iii)} $\bm{ A_1}$ produces result \textbf{faster than} $\bm{ A_2}$ when $\bm{ n < 20}$
\\By substitution different values less than 20, \\Ex: for n = 10, $A_1$ takes 0.1024 seconds and $A_2$ takes 10 seconds\\Ex: for n = 19, $A_1$ takes 52.4288 seconds and $A_2$ takes 68.59 seconds
\section*{Question 3}
 Prove or disprove the following with valid arguments:\\\\(i) $3n^{2}+ 1000 =O(n)$.
\\(ii) $2n^{3}\log ({n}) = \Theta(n^{3})$.
\\(iii) $3^{n}n^{4}+ 8 * 4^{n}n^{3}=O(3^{n}n^{4})$.
\section*{Solution:}
\subsection*{(i)} \textbf{False}.Disprove by contradiction.\\ Let $f(n) = 3n^{2} + 1000$ and $g(n) = n$ \\ if $f(n) = O(g(n))$ then $f(n) \leq C * g(n)$ such that $n>n_0$ for a $C>0$
	\\But since for any c:(such that $n>c$)$$3n^{2} > cn$$  Hence, $\bm{3n^{2} + 1000 \neq O(n)}$\\\\\\
\subsection*{(ii)} \textbf{False}. Disprove by contradiction.
\\Let $f(n) = 2n^{3} \log (n)$ \\ Let $g(n) = n^{3}$
\\ if $2n^{3}\log ({n}) = \Theta(n^{3})$ then, there should exist $c_1, c_2 >0$ and positive $n>n_0$ such that, $$ c_1 g(n) \leq f(n) \leq c_2 g(n) $$ i.e., $f(n)$ is $O(g(n))$ and $f(n)$ is $\Omega(g(n))$
\\Assuming, $f(n)$ is $O(g(n))$, then $$2n^{3} \log (n) \leq C * n^{3}$$ $$2 \log (n) \leq C $$ $$\log (n) \leq C' $$ The above can not be true since c must be a constant but $\log (n)$ is unbounded. This is a contradiction with the assumption that we can find such a constant c. Therefore, $2n^{3}\log ({n})$ is not $O(n^{3})$ and hence, $\bm{2n^{3}\log ({n}) \neq \Theta(n^{3})}$
\subsection*{(iii)} \textbf{False}. Disprove by contradiction.
\\ Let $f(n) = 3^{n}n^{4} + 8 * 4^{n}n^{3}$ and $g(n) = 3^{n}n^{4}$ \\ if $f(n) = O(g(n))$ then $f(n) \leq C * g(n)$ such that $n>n_0$ for a $C>0$
	\\Assuming, $f(n)$ is $O(g(n))$, then $$ 3^{n}n^{4} + 8 * 4^{n}n^{3} \leq C * 3^{n}n^{4} $$ $$ 8 * 4^{n}n^{3} \leq  3^{n}n^{4}(C - 1)$$ $$\frac{8 * 4^{n}}{3^{n}n} \leq C - 1 $$ $$ \frac{8 * 4^{n}}{3^{n}n} + 1 \leq C $$
The above can not be true since c must be a constant but $ \frac{8 * 4^{n}}{3^{n}n}$ is unbounded. This is a contradiction with the assumption that we can find such a constant c. Therefore, $\bm{3^{n}n^{4} + 8 * 4^{n}n^{3} \neq O(3^{n}n^{4})}$\\\\\\\\



\section*{Question 4}
Take the following list of functions and arrange them in ascending order of growth rate.  That is, if function $g(n)$ immediately follows $f(n)$ in your list, then it should be the case that $f(n)$ is $O(g(n))$.\\
\\(i)$f_1(n) =n^{4.2}$.
\\(ii)$f_2(n) = (2n)^{1.2}$.
\\(iii)$f_3(n) =n^{4.1}+ 87$.
\\(iv)$f_4(n) = 60^{n}$.
\\(v)$f_5(n) = 180^{n}$.
\section*{Solution:}
$\bm{f_2(n) < f_3(n) < f_1(n) < f_4(n) < f_5(n)}$
\\Since $f_4$ and $f_5$ are exponential and will grow the fastest we'll put them at the end of the list. $f_4 < f_5$ because $60 < 180$. Other three functions are polynomial and will grower slower then exponential. We can represent $f_1$, $f_2$ and $f_3$ as: $$ f_1(n) =n^{4.2} = n^{4} \times n^{0.2} $$ $$ f_2(n) =(2n)^{1.2} = 2n \times (2n)^{0.2} $$
 $$ f_3(n) =n^{4.1} + 87 = n^{4} \times n^{0.1} + 87$$ The polynomials with higher degree grows faster. hence, $f_1, f_3 > f_2$. Since $n \to \infty$, 87 in $f_3$ is negligible. Hence, $f_1 > f_3$




\end{document}